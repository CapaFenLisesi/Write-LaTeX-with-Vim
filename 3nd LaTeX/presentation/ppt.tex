\documentclass{beamer}
\mode<presentation>
{
  \usetheme{Warsaw}
  % or ...

  %\setbeamercovered{transparent}
  % or whatever (possibly just delete it)
}


\usepackage[english]{babel}
% or whatever

\usepackage[latin1]{inputenc}
% or whatever

\usepackage{times}
\usepackage[T1]{fontenc}
% Or whatever. Note that the encoding and the font should match. If T1
% does not look nice, try deleting the line with the fontenc.
\usepackage{listings}
\lstset{basicstyle=\ttfamily}
\title[How to start \LaTeX ing] % (optional, use only with long paper titles)
{How to start \LaTeX ing}

\subtitle
{In my opinion...} % (optional)

\author[taper] % (optional, use only with lots of authors)
{we.taper\inst{1}}

\institute[Universities of Somewhere and Elsewhere] % (optional, but mostly needed)
{
  \inst{1}%
  Physics, SUSTech
}

\subject{Talks}
% This is only inserted into the PDF information catalog. Can be left
% out. 

\pgfdeclareimage[height=3cm]{university-logo}{sustech-logo.png}
\logo{\pgfuseimage{university-logo}}



% Delete this, if you do not want the table of contents to pop up at
% the beginning of each subsection:
\AtBeginSubsection[]
{
  \begin{frame}<beamer>{Outline}
    \tableofcontents[currentsection,currentsubsection]
  \end{frame}
}


% If you wish to uncover everything in a step-wise fashion, uncomment
% the following command: 
%\beamerdefaultoverlayspecification{<+->}


\begin{document}

\begin{frame}
  \titlepage
\end{frame}

\begin{frame}{Outline}
  \tableofcontents
  % You might wish to add the option [pausesections]
\end{frame}


% Since this a solution template for a generic talk, very little can
% be said about how it should be structured. However, the talk length
% of between 15min and 45min and the theme suggest that you stick to
% the following rules:  

% - Exactly two or three sections (other than the summary).
% - At *most* three subsections per section.
% - Talk about 30s to 2min per frame. So there should be between about
%   15 and 30 frames, all told.

\section{Start with Examples}

%\subsection[Basic Structure]{Basic Structure of \LaTeX}
\begin{frame}[fragile]{Basic Structure of \LaTeX}
    \begin{minipage}[c]{0.6\textwidth}
    \begin{itemize}
    \item 
    Basic structure of a \LaTeX file looks like this $ \Longrightarrow$
    \item Comment starts with \%
    \item Preamble: puts everything that one do not expect to be seen on a
        document. (Compare: \texttt{import} in Python, Java.)
    \item 
    Packages and Classes. 
    \begin{itemize}
        \item Packages = tools to create symbols. 
        \item Classes = Templates in Word, Structure of the document.
    \end{itemize}
    \end{itemize}
    \end{minipage}
    \begin{minipage}[b]{0.3\textwidth}
    \begin{lstlisting}[language=tex]
\documentclass{...}
% Preamble
\begin{document}
...
\end{document}
    \end{lstlisting}
    \end{minipage} 
\end{frame}

%\subsection[Beamer Class]{First Example: Beamer}

\begin{frame}{Beamer}
  % - A title should summarize the slide in an understandable fashion
  %   for anyone how does not follow everything on the slide itself.

  \begin{itemize}
  \item
      Available online:
      \href{https://github.com/josephwright/beamer}{\beamergotobutton{Github
      Beamer}}
      \pause
  \item Open and Compile
      \pause
  \item Read the comments and modifies.
  \item Again and again.
  \end{itemize}
\end{frame}

\section{Programming Jargon}

%\subsection{Compile Procedure}
\begin{frame}[fragile]{Compile Procedure}
    \begin{itemize}
        \item C:
        \begin{itemize}
            \item Code (*.h) $\to$ linking (*.o) $\to$ Program (*.exe)
        \end{itemize}
        \item \LaTeX
        \begin{itemize}
            \item Code (*.tex) $\to$ Auxiliary file $\to$ PDF (*.pdf)
            \item Compiler: \LaTeX, Xe\LaTeX, Lua\LaTeX.
        \end{itemize}
        \pause
        \item Understanding \textbf{PATH}
        \begin{itemize}
            \item Relative v.s. Absolute path
            \item \texttt{PATH} variable
        \end{itemize}
    \end{itemize}
\end{frame}

%\subsection{Read Commands}
\begin{frame}[fragile]{Read Commands}
    \begin{itemize}
        \item Understand Meanings:
            \begin{lstlisting}[language=tex]
                \textbackslash \phantom
            \end{lstlisting}
        \pause
        \item Guess Abbreviations
        \begin{itemize}
            \item Msg
            \item Ctrl, Prt, Bf, Err\dots
            \item \textbackslash vskip, \textbackslash vphantom.
            \item Bib (bibliography)
        \end{itemize}
    \end{itemize}
\end{frame}

%\subsection{Err Msg}
\begin{frame}{Err Msg}
    \begin{enumerate}
        \item Always Write \& Compile \& Check
        \item Line number is important information, but is occasionally useless.
        \item First Error is the most relevant one.
        \item Undefined Control Sequence
            \pause
        \item Err msg is sometimes unreliable, READ your code!
        \item Produce \textbf{Minimal Working File} with Commenting! 
            \begin{itemize}
            \item Ctrl+T on TeX Studio
            \end{itemize}
    \end{enumerate}
\end{frame}

%\subsection{Indent, Spaces and Reserved Keywords}
\begin{frame}{Indent, Spaces and Reserved Keywords}
    \begin{itemize}
        \item<1-> Good indentation means, everything\dots ,IMO.
        \begin{itemize}
            \item Structure: begin, end.
            \item Align curly-braces.
        \end{itemize}
        \item<2-> \textbf{SPACES \& CHANGE LINE}!
            \begin{itemize}
                \item \LaTeX{} treat spaces as would any Programming language
                    do:

                    Ignores multiples of them.
                \item Read line number and hit \beamergotobutton{Enter}
                    whenever convenient.
                \item Good line number + good spaces = fewer bugs
            \end{itemize}
        \item<3-> Reserved Keywords:
            \begin{itemize}
                \item Everything starts with \textbackslash
                \item \textbackslash\textbackslash, \&, Non-ASCII
            \end{itemize}
    \end{itemize}
\end{frame}

\begin{frame}[t]{Programmer Tools}
    \begin{itemize}
        \item<1-> IDE: TeX Studio, Vim.
        \begin{itemize}
        \item<2-> \textbf{Snippets}: TeX Studio, Vim, AutoHotkey, etc.
        \end{itemize}
        \item<1-> Source Code Management: Git
        \item<1-> Online Editor: ShareLatex and Overleaf.
    \end{itemize}
\end{frame}

\begin{frame}[fragile]{\LaTeX{} Sucks}
    \begin{itemize}
        \item Bad Programmer writing \LaTeX{} codes.
        \item Everything expects an argument: 

            \LaTeX s 
            \begin{lstlisting}[language=tex]
                \LaTeX s
            \end{lstlisting}
            is different from

            \LaTeX{} s.
            \begin{lstlisting}[language=tex]
                \LaTeX{} s
            \end{lstlisting}
        \item Compile it \textbf{AGAIN}, esp. containing bibliography files.
        \item Horrible Auxiliary files: delete them when there is no Minimal
            working file.
    \end{itemize}
\end{frame}

\section{Getting Resources}
\begin{frame}{Getting Resources}
    \begin{itemize}
        \item Google is always your best choice.
            \begin{itemize}
                \item And information on WikiBooks, StackExchange \& ShareLatex
                    are excellent.
            \end{itemize}
            \begin{itemize}
                \item E.g.: "Quotation mark" in \LaTeX.
            \end{itemize}
        \item \texttt{CTAN}: Comprehensive \TeX{} Archive Network, containing
            almost all possible packages.
            \begin{itemize}
                \item E.g.: Physics Packages
            \end{itemize}
        \item GitHub: modern codes online
            \begin{itemize}
                \item E.g.: Beamer on GitHub.
            \end{itemize}
    \end{itemize}
\end{frame}

\section{End}
\begin{frame}{The End}
    \begin{center}
    What do u have in mind?
    \end{center}
\end{frame}
\end{document}


